\documentclass[aspectratio=169]{beamer}

\usepackage[utf8]{inputenc}
\usepackage{soul}
\usepackage{pdfpcnotes}
\usepackage{listings}
\usepackage{tikz}
\usepackage{booktabs}
\usepackage{minted}
\usepackage[ngerman]{babel}

\usetheme{Hannover}
\usecolortheme{dove}

\AtBeginSection[]{
  \begin{frame}
  \vfill
  \centering
  \begin{beamercolorbox}[sep=8pt,center,shadow=true,rounded=true]{title}
    \usebeamerfont{title}\insertsectionhead\par%
  \end{beamercolorbox}
  \vfill
  \end{frame}
}

\lstset{language=C++,
                basicstyle=\ttfamily,
                keywordstyle=\color{blue}\ttfamily,
                stringstyle=\color{red}\ttfamily,
                commentstyle=\color{purple}\ttfamily,
                morecomment=[l][\color{magenta}]{\#}
}


\title{Eine Einführung in Git}
\author{Paul Nykiel}
\date{\today}

\begin{document}
\maketitle

\frame{
    \pnote{Sofort Fragen!}
    \pnote{Feedback erwünscht}
    \tableofcontents
}

\section{Einleitung}
\begin{frame}
    \frametitle{Wofür ein Versionsverwaltungssystem?}
    Situation: Mehrere Leute arbeiten über längere Zeit an einer Codebase

    \vspace{.5cm}

    \pause
    Probleme:
    \pause
    \begin{itemize}
        \item Datei Austausch zwischen Nutzer
            \pnote{Dateien müssen für alle Nutzer in aktueller Form vorliegen}
            \pause
        \item Aber: nicht sofort, erst wenn fertig
            \pnote{Vgl. Google-Docs, wenn ein User was ändert compiliert Code nicht}
            \pause
        \item Alte Codestände sollten getestet werden können
            \pnote{Ermöglicht es zu testen "war das schon immer so"?}
            \pause
    \end{itemize}
\end{frame}

\begin{frame}
    \frametitle{Was noch?}
    \begin{itemize}
        \item Authentisierung: von wem ist der Code
            \pnote{Hilft bei Fragen}
            \pause
        \item Kein permanenter Internetzugriff 
            \pnote{Damit immer gearbeitet werden kann}
            \pause
        \item Einfache Nutzung
            \pnote{Nur Mittel zum Zweck}
            \pause
        \item Schnell
            \pnote{Arbeiten sollte nicht behindert werden}
            \pause
        \item Sicher
            \pnote{Nur berechtigte Nutzer sollten Code ändern dürfen}
    \end{itemize}
\end{frame}

\begin{frame}
    \frametitle{Git}
    \begin{columns}
        \begin{column}{.7\textwidth}
            \begin{itemize}
                \item<1-> Freie Software zur Versionsverwaltung
                    \pnote{Frei: FOSS und Beer}
                \item<2-> Dezentral
                    \pnote{Es kann aber auch eine Server geben}
                \item<3-> Wurde 2005 von Linus Torvals für Linux initiiert
                    \pnote{Sehr großes, sehr verteiltes Softwareprojekt}
                \item<4-> Defakto Standard
                    \pnote{Wird quasi überall genutzt}
            \end{itemize}
        \end{column}
        \begin{column}{.3\textwidth}
            \includegraphics[width=\textwidth]{Git-Icon-1788C.eps}
        \end{column}
    \end{columns}
\end{frame}

\section{Konzept}
\begin{frame}
    \frametitle{Aufbau Git}
    \begin{itemize}
        \item Repository: Eine zusammenhänge Codebase/Projekt
            \pnote{Muss nicht zwingend Code sein!}
            \pnote{Größte Einheit}
            \pause
        \item Ein Repository ist ein (azyklischer) gerichterer Graph aus Commits
            \pnote{Jedes Repository besteht aus Commits}
            \pause
    \end{itemize}

    \includegraphics[width=\textwidth]{graph.pdf}
\end{frame}

\begin{frame}[fragile]
    \frametitle{Commit}
    \begin{columns}
        \begin{column}{.25\textwidth}
            \begin{itemize}
                \item Code-Stand
                    \pnote{Eigentlicher Inhalt}
                    \pause
                \item Eindeutige Bezeichnung
                    \pnote{SHA-1 Hash der Änderungen um eindeutig zuordnen zu können}
                    \pause
                \item Autor
                    \pnote{Siehe Authentisierung oben}
                    \pause
                \item Datum
                    \pause
                \item Nachricht
                    \pnote{Text vom Autor, was wurde geändert}
            \end{itemize}
        \end{column}
        \begin{column}{.75\textwidth}
            \begin{verbatim}
commit feeecb67fe1fa0490a2b836d5ba35da5812a3d27
Author: Paul Nykiel
Date:   Mon Apr 20 22:55:41 2020 +0200

Added ADTF type header
            \end{verbatim}
        \end{column}
    \end{columns}
\end{frame}

\begin{frame}
    \frametitle{Einen Commit anlegen}
    \begin{center}
        \only<1>{
    \begin{tikzpicture}
        \filldraw[draw=black, fill=gray!50] (0,0) rectangle (3,3);

        \node[above]() at (1.5, 3){Working-Dir};

        \filldraw[draw=black, fill=gray!10] (0.25,2.25) rectangle (2.75,2.75) node() [pos=.5] {File 1};
        \filldraw[draw=black, fill=gray!10] (0.25,1.25) rectangle (2.75,1.75) node() [pos=.5] {File 2};
        \filldraw[draw=black, fill=gray!10] (0.25,0.25) rectangle (2.75,0.75) node() [pos=.5] {File 3};
        \node() at(13,4){};
    \end{tikzpicture}
}

\only<2>{
    \begin{tikzpicture}
        \filldraw[draw=black, fill=gray!50] (0,0) rectangle (3,3);
        \filldraw[draw=black, fill=gray!50] (5,0) rectangle (8,3);

        \node[above]() at (1.5, 3){Working-Dir};
        \node[above]() at (6.5, 3){Staging};

        \draw[line width=1pt, ->] (3, 1.5) -- (5,1.5) node[midway, above]() {Add};

        \filldraw[draw=black, fill=gray!10] (0.25,2.25) rectangle (2.75,2.75) node() [pos=.5] {File 1};
        \filldraw[draw=black, fill=gray!10] (0.25,1.25) rectangle (2.75,1.75) node() [pos=.5] {File 2};
        \filldraw[draw=black, fill=gray!10] (0.25,0.25) rectangle (2.75,0.75) node() [pos=.5] {File 3};
        \node() at(13,4){};
    \end{tikzpicture}
}

\only<3>{
    \begin{tikzpicture}
        \filldraw[draw=black, fill=gray!50] (0,0) rectangle (3,3);
        \filldraw[draw=black, fill=gray!50] (5,0) rectangle (8,3);

        \node[above]() at (1.5, 3){Working-Dir};
        \node[above]() at (6.5, 3){Staging};

        \draw[line width=1pt, ->] (3, 1.5) -- (5,1.5) node[midway, above]() {Add};

        \filldraw[draw=black, fill=gray!10] (0.25,0.25) rectangle (2.75,0.75) node() [pos=.5] {File 3};
        \filldraw[draw=black, fill=gray!10] (0.25,1.25) rectangle (2.75,1.75) node() [pos=.5] {File 2};
        \filldraw[draw=black, fill=gray!10] (5.25,0.25) rectangle (7.75,0.75) node() [pos=.5] {File 1};
        \node() at(13,4){};
    \end{tikzpicture}
}

\only<4>{
    \begin{tikzpicture}
        \filldraw[draw=black, fill=gray!50] (0,0) rectangle (3,3);
        \filldraw[draw=black, fill=gray!50] (5,0) rectangle (8,3);

        \node[above]() at (1.5, 3){Working-Dir};
        \node[above]() at (6.5, 3){Staging};

        \draw[line width=1pt, ->] (3, 1.5) -- (5,1.5) node[midway, above]() {Add};

        \filldraw[draw=black, fill=gray!10] (0.25,0.25) rectangle (2.75,0.75) node() [pos=.5] {File 3};
        \filldraw[draw=black, fill=gray!10] (5.25,1.25) rectangle (7.75,1.75) node() [pos=.5] {File 2};
        \filldraw[draw=black, fill=gray!10] (5.25,0.25) rectangle (7.75,0.75) node() [pos=.5] {File 1};
        \node() at(13,4){};
    \end{tikzpicture}
}

\only<5>{
    \begin{tikzpicture}
        \filldraw[draw=black, fill=gray!50] (0,0) rectangle (3,3);
        \filldraw[draw=black, fill=gray!50] (5,0) rectangle (8,3);
        \filldraw[draw=black, fill=gray!50] (10,0) rectangle (13,3);

        \node[above]() at (1.5, 3){Working-Dir};
        \node[above]() at (6.5, 3){Staging};
        \node[above]() at (11.5, 3){Graph};

        \draw[line width=1pt, ->] (3, 1.5) -- (5,1.5) node[midway, above]() {Add};
        \draw[line width=1pt, ->] (8, 1.5) -- (10,1.5) node[midway, above]() {Commit};

        \filldraw[draw=black, fill=gray!10] (0.25,0.25) rectangle (2.75,0.75) node() [pos=.5] {File 3};
        \filldraw[draw=black, fill=gray!10] (5.25,1.25) rectangle (7.75,1.75) node() [pos=.5] {File 2};
        \filldraw[draw=black, fill=gray!10] (5.25,0.25) rectangle (7.75,0.75) node() [pos=.5] {File 1};
        \node() at(13,4){};
    \end{tikzpicture}
}

\only<6>{
    \begin{tikzpicture}
        \filldraw[draw=black, fill=gray!50] (0,0) rectangle (3,3);
        \filldraw[draw=black, fill=gray!50] (5,0) rectangle (8,3);
        \filldraw[draw=black, fill=gray!50] (10,0) rectangle (13,3);

        \node[above]() at (1.5, 3){Working-Dir};
        \node[above]() at (6.5, 3){Staging};
        \node[above]() at (11.5, 3){Graph};

        \draw[line width=1pt, ->] (3, 1.5) -- (5,1.5) node[midway, above]() {Add};
        \draw[line width=1pt, ->] (8, 1.5) -- (10,1.5) node[midway, above]() {Commit};

        \filldraw[draw=black, fill=gray!10] (0.25,0.25) rectangle (2.75,0.75) node() [pos=.5] {File 3};
        \filldraw[draw=black, fill=gray!10] (5.25,0.25) rectangle (7.75,0.75) node() [pos=.5] {File 1};
        \filldraw[draw=black, fill=gray!10] (10.25,0.25) rectangle (12.75,0.75) node() [pos=.5] {File 2};
        \node() at(13,4){};
    \end{tikzpicture}
}

    \end{center}
\end{frame}

\begin{frame}
    \frametitle{Remotes}
    \begin{columns}
        \begin{column}{.5\textwidth}
        \begin{itemize}
            \item<1-> Bis jetzt alles lokal
                \pnote{Aber primäre Änderung war verteiltes Arbeiten}
            \item<2-> Codebase auf anderem Host: \glqq{}Remote\grqq{}
                \pnote{Andere Hostname wird über Netzwerkadresse und Pfad indentifiziert}
            \item<3-> Operationen: Commits von Remote kopieren, Commits zu Remote kopieren
                \pnote{Nur Commits und Teile vom Graphen können übertragen werden, nicht das aktuelle Workingdirectory}
            \item<4-> Oftmals ein zentraler Server
                \pnote{Wird oftmals origin genannt}
        \end{itemize}
        \end{column}
        \begin{column}{.5\textwidth}
            \begin{center}
                \begin{tikzpicture}
                    \filldraw[draw=black, fill=gray!50] (0,0) rectangle (2,3);
                    \filldraw[draw=black, fill=gray!50] (4,0) rectangle (6,3);

                    \draw[line width=1pt,->] (2,2) -- (4,2) node[midway, above]() {Push};
                    \draw[line width=1pt,->] (4,1) -- (2,1) node[midway, below]() {Pull};

                    \node[above]() at (1,3) {Local};
                    \node[above]() at (5,3) {Remote};
                \end{tikzpicture}
            \end{center}
        \end{column}
    \end{columns}
\end{frame}

\begin{frame}
    \frametitle{Branches}
    \begin{center}
        \includegraphics[width=.85\textwidth]{branch.pdf}
    \end{center}
    \pause
    {\Large Ein Branch ist eine Liste von Commits}
\end{frame}

\section{Nutzung}
\begin{frame}
    \frametitle{Überblick}
    \begin{itemize}
        \item Manipulation des Zustandes über Git (\url{https://git-scm.com/})
            \pnote{Einzelne Executable über Shell}
            \pause
        \item Nutzung: \lstinline{git command arguments}
            \pnote{command ist grobe beschreibung welche Aktion, arguments für details}
            \pause
        \item Häufige Befehle:
            \pnote{Auflistung der wichtigsten Befehle, also Nutzung immer git command}
            \pause
            \begin{itemize}
                \item \lstinline{add}
                    \pnote{Dateien zur Staging Area hinzufügen}
                    \pause
                \item \lstinline{commit}
                    \pnote{Commit aus Dateien in Staging generieren}
                    \pause
                \item \lstinline{push}
                    \pnote{Lokalen (Teil-) Graph auf Remote übertragen}
                    \pause
                \item \lstinline{pull}
                    \pnote{(Teil-) Graph zu lokal übertragen}
                    \pause
                \item \lstinline{checkout}
                    \pnote{Anderen Knoten in Graph auswählen}
                    \pause
                \item \lstinline{status}
                    \pnote{Zustand des Repositories anzeigen}
                    \pause
                \item \lstinline{log}
                    \pnote{Letzte Commits anzeigen}
            \end{itemize}
    \end{itemize}
\end{frame}

\begin{frame}
    \frametitle{add}
    \begin{center}
        {\Large \lstinline{git add FILES...}}
        \pause
    
        \vspace{1cm}

        {\Large \lstinline{git add -A}}
        \pause
    
        \vspace{1cm}

        {\Large \lstinline{git add -u}}
    \end{center}
\end{frame}

\begin{frame}
    \frametitle{Commit}
    \begin{center}
        {\Large \lstinline{git commit -m "message"}}
    \end{center}
\end{frame}

\begin{frame}
    \frametitle{Push}
    \begin{center}
        {\Large \lstinline{git push REMOTE BRANCH}}
    \end{center}
\end{frame}

\begin{frame}
    \frametitle{Pull}
    \begin{center}
        {\Large \lstinline{git pull REMOTE BRANCH}}
    \end{center}
\end{frame}

\begin{frame}
    \frametitle{Checkout}
    \begin{center}
        {\Large \lstinline{git checkout IDENTIFIER}}

        mit Identifier: Branch Name, Commit-Hash, \ldots
        \pause

        \vspace{1cm}
    
        {\Large \lstinline{git checkout -b NEW\_BRANCH}}
    \end{center}
\end{frame}

\begin{frame}
    \frametitle{Information}
    \begin{center}
        {\Large \lstinline{git status}}
        \pause

        \vspace{1cm}

        {\Large \lstinline{git log}}
    \end{center}
\end{frame}

\section{GitLab}

\section{Git bei uns}

\section{Abschluss}

\section{Praxis}


\end{document}
